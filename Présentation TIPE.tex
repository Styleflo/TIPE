% Preamble
\documentclass[t,10pt]{beamer}

% Packages
\usepackage{amsmath}
\usepackage[T1]{fontenc}
\usepackage[frenchb]{babel}
\usepackage{pslatex}
\usepackage{romanbar}
\usepackage{graphicx}
\usepackage{tikz}


\usetheme{Warsaw}

% Document
\title{Le routage réseau des télécommunications mobiles }
\subtitle{TIPE-2022/2023}
\author{Florian Touraine -13175}
\date{}

\begin{document}
    \maketitle

    %Sommaire
    \begin{frame}
        \tableofcontents[]
    \end{frame}

    \section{Présentation du Sujet}
    \subsection{Problématique et Motivations}
    \begin{frame}\frametitle{\underline{Présentation du Sujet} \\ \small Problématique et Motivations}
        \underline{Problématique:} \\
        Comment router au mieux les chemins des communications entre antennes ?
        \\
        \vspace{0.5cm}
        \underline{L'étude de ce problème est motivée par:}
        \begin{enumerate}
            \item La croissance de la population et donc des réseaux.
            \item La volonté d'éviter les saturations et les surcharges du réseau.
        \end{enumerate}
    \end{frame}

    \subsection{Exemple avec Paris}
    \begin{frame}\frametitle{\small Représentation de Paris}
        \underline{Répartition des antennes dans la ville de Paris:}
        \begin{enumerate}
            \item Nombre d'Antennes : 1000
            \item Rayon de Paris : 10 km
        \end{enumerate}
        \includegraphics[scale=0.5]{Image antennes de Paris}
    \end{frame}

    \subsection{Exemple à 9 antennes}
    \begin{frame}\frametitle{\small Exemple à 9 antennes}
        \underline{On declare le type antennes:}
        \\
        \vspace{0.1cm}
         \includegraphics[scale=0.4]{Image classe}
        \\
        \underline{on se donne les 9 antennes selon ce type:}
        \\
        \vspace{0.1cm}
        \includegraphics[scale=0.4]{Image antennes}
    \end{frame}

    \begin{frame}
        \includegraphics[scale=0.6]{Image à 9 antennes}
    \end{frame}

    \begin{frame}
        \includegraphics[scale=0.6]{Image à 9 antennes reliées}
    \end{frame}

    \begin{frame}
        \vspace{1.42cm}
        \begin{tikzpicture}
            \path
            (0,0) node[draw,shape=circle](0){0}
            (1,1) node[draw,shape=circle](1){1}
            (1,-1) node[draw,shape=circle](2){2}
            (-1,-1) node[draw,shape=circle](3){3}
            (-1,1) node[draw,shape=circle](4){4}
            (2,0) node[draw,shape=circle](5){5}
            (0,-2) node[draw,shape=circle](6){6}
            (-2,0) node[draw,shape=circle](7){7}
            (0,2) node[draw,shape=circle](8){8}
            \footnotesize
            \draw(8)--(1)node[midway,above,sloped]{$1000$};
            \draw(8)--(4)node[midway,above,sloped]{$1000$};
            \draw(4)--(0)node[midway,above,sloped]{$100$};
            \draw(1)--(0)node[midway,above,sloped]{$100$};
            \draw(4)--(7)node[midway,above,sloped]{$1000$};
            \draw(1)--(5)node[midway,above,sloped]{$100$};
            \draw(7)--(3)node[midway,above,sloped]{$100$};
            \draw(0)--(3)node[midway,above,sloped]{$100$};
            \draw(0)--(2)node[midway,above,sloped]{$100$};
            \draw(5)--(2)node[midway,above,sloped]{$100$};
            \draw(3)--(6)node[midway,above,sloped]{$100$};
            \draw(2)--(6)node[midway,above,sloped]{$10000$};
        \end{tikzpicture}
    \end{frame}

    \begin{frame}
        \vspace{1cm}
    \begin{minipage}{0.48\linewidth}
        \begin{tikzpicture}
            \path
            (0,2.5) node(nb){1}
            (0,-2.5) node(nb2){1}
            (0,0) node[draw,shape=circle](0){0}
            (1,1) node[draw,shape=circle](1){1}
            (1,-1) node[draw,shape=circle](2){2}
            (-1,-1) node[draw,shape=circle](3){3}
            (-1,1) node[draw,shape=circle](4){4}
            (2,0) node[draw,shape=circle](5){5}
            (0,-2) node[draw,shape=circle](6){6}
            (-2,0) node[draw,shape=circle](7){7}
            (0,2) node[draw,shape=circle](8){8}
            \footnotesize
            \draw(8)--(1)node[midway,above,sloped]{$900$};
            \draw[red, <->, =>latex](8)--(4)node[midway,above,sloped]{$900$};
            \draw(4)--(0)node[midway,above,sloped]{$0$};
            \draw(1)--(0)node[midway,above,sloped]{$100$};
            \draw[red, <->, =>latex](4)--(7)node[midway,above,sloped]{$900$};
            \draw(1)--(5)node[midway,above,sloped]{$100$};
            \draw[red, <->, =>latex](7)--(3)node[midway,above,sloped]{$0$};
            \draw(0)--(3)node[midway,above,sloped]{$0$};
            \draw(0)--(2)node[midway,above,sloped]{$100$};
            \draw(5)--(2)node[midway,above,sloped]{$100$};
            \draw[red, <->, =>latex](3)--(6)node[midway,above,sloped]{$0$};
            \draw(2)--(6)node[midway,above,sloped]{$10000$};
        \end{tikzpicture}
    \end{minipage}
    \hfill
    \begin{minipage}{0.48\linewidth}
        \begin{enumerate}
            \item Une personne située au sommet 8 communique avec une autre au sommet 6.
            \item Elle utilise la totalité de la bande passante disponible sur le chemin (100) (= bonne communication).
            \item Elle passe par le sommet 4 et utilise la totalité de sa bande passante
            donc l'arrete 4 -> 0 est elle aussi diminuée.
        \end{enumerate}
    \end{minipage}
    \end{frame}

    \begin{frame}
        \vspace{1cm}
    \begin{minipage}{0.48\linewidth}
        \begin{tikzpicture}
            \path
            (0,2.5) node(nb){4}
            (0,-2.5) node(nb2){4}
            (0,0) node[draw,shape=circle](0){0}
            (1,1) node[draw,shape=circle](1){1}
            (1,-1) node[draw,shape=circle](2){2}
            (-1,-1) node[draw,shape=circle](3){3}
            (-1,1) node[draw,shape=circle](4){4}
            (2,0) node[draw,shape=circle](5){5}
            (0,-2) node[draw,shape=circle](6){6}
            (-2,0) node[draw,shape=circle](7){7}
            (0,2) node[draw,shape=circle](8){8}
            \footnotesize
            \draw(8)--(1)node[midway,above,sloped]{$900$};
            \draw[red, <->, =>latex](8)--(4)node[midway,above,sloped]{$900$};
            \draw(4)--(0)node[midway,above,sloped]{$0$};
            \draw(1)--(0)node[midway,above,sloped]{$100$};
            \draw[red, <->, =>latex](4)--(7)node[midway,above,sloped]{$900$};
            \draw(1)--(5)node[midway,above,sloped]{$100$};
            \draw[red, <->, =>latex](7)--(3)node[midway,above,sloped]{$0$};
            \draw(0)--(3)node[midway,above,sloped]{$0$};
            \draw(0)--(2)node[midway,above,sloped]{$100$};
            \draw(5)--(2)node[midway,above,sloped]{$100$};
            \draw[red, <->, =>latex](3)--(6)node[midway,above,sloped]{$0$};
            \draw(2)--(6)node[midway,above,sloped]{$10000$};
        \end{tikzpicture}
    \end{minipage}
    \hfill
    \begin{minipage}{0.48\linewidth}
        \begin{enumerate}
            \item Quatre personnes situées au sommet 8 communique avec quatre autre au sommet 6.
            \item Elles se repartissent la bande passante équitablement (25 par personnes).
            \item Mauvaise communication et mauvaise répartition.
        \end{enumerate}
    \end{minipage}
    \end{frame}

    \begin{frame}
        \vspace{1cm}
    \begin{minipage}{0.48\linewidth}
        \begin{tikzpicture}
            \path
            (0,2.5) node(nb){4}
            (0,-2.5) node(nb2){4}
            (0,0) node[draw,shape=circle](0){0}
            (1,1) node[draw,shape=circle](1){1}
            (1,-1) node[draw,shape=circle](2){2}
            (-1,-1) node[draw,shape=circle](3){3}
            (-1,1) node[draw,shape=circle](4){4}
            (2,0) node[draw,shape=circle](5){5}
            (0,-2) node[draw,shape=circle](6){6}
            (-2,0) node[draw,shape=circle](7){7}
            (0,2) node[draw,shape=circle](8){8}
            \footnotesize
            \draw[blue, <->, =>latex](8)--(1)node[midway,above,sloped]{$800$};
            \draw[red, <->, =>latex](8)--(4)node[midway,above,sloped]{$800$};
            \draw[brown, <->, =>latex](4)--(0)node[midway,above,sloped]{$0$};
            \draw[orange, <->, =>latex](1)--(0)node[midway,above,sloped]{$0$};
            \draw[red, <->, =>latex](4)--(7)node[midway,above,sloped]{$900$};
            \draw[blue, <->, =>latex](1)--(5)node[midway,above,sloped]{$900$};
            \draw[red, <->, =>latex](7)--(3)node[midway,above,sloped]{$0$};
            \draw[brown, <->, =>latex](0)--(3)node[midway,above,sloped]{$0$};
            \draw[orange, <->, =>latex](0)--(2)node[midway,above,sloped]{$0$};
            \draw[blue, <->, =>latex](5)--(2)node[midway,above,sloped]{$900$};
            \draw[red, <->, =>latex](3)--(6)node[midway,above,sloped]{$0$};
            \draw[blue, <->, =>latex](2)--(6)node[midway,above,sloped]{$9900$};
        \end{tikzpicture}
    \end{minipage}
    \hfill
    \begin{minipage}{0.48\linewidth}
        \begin{enumerate}
            \item Les chemins sont mieux répartis.
            \item Pour chaque utilisateur, 50 de bande passante leur est alloué.
        \end{enumerate}
    \end{minipage}
    \end{frame}

    \begin{frame}
        \vspace{1cm}
        \begin{minipage}{0.48\linewidth}
            \begin{tikzpicture}
                \path
                (0,2.5) node(nb){4}
                (0,-2.5) node(nb2){4}
                (0,0) node[draw,shape=circle](0){0}
                (1,1) node[draw,shape=circle](1){1}
                (1,-1) node[draw,shape=circle](2){2}
                (-1,-1) node[draw,shape=circle](3){3}
                (-1,1) node[draw,shape=circle](4){4}
                (2,0) node[draw,shape=circle](5){5}
                (0,-2) node[draw,shape=circle](6){6}
                (-2,0) node[draw,shape=circle](7){7}
                (0,2) node[draw,shape=circle](8){8}
                \footnotesize
                \draw(8)--(1)node[midway,above,sloped]{$100$};
                \draw(8)--(4)node[midway,above,sloped]{$100$};
                \draw(4)--(0)node[midway,above,sloped]{$50$};
                \draw(1)--(0)node[midway,above,sloped]{$50$};
                \draw(4)--(7)node[midway,above,sloped]{$50$};
                \draw(1)--(5)node[midway,above,sloped]{$50$};
                \draw(7)--(3)node[midway,above,sloped]{$50$};
                \draw(0)--(3)node[midway,above,sloped]{$50$};
                \draw(0)--(2)node[midway,above,sloped]{$50$};
                \draw(5)--(2)node[midway,above,sloped]{$50$};
                \draw(3)--(6)node[midway,above,sloped]{$100$};
                \draw(2)--(6)node[midway,above,sloped]{$100$};
            \end{tikzpicture}
        \end{minipage}
        \hfill
        \begin{minipage}{0.48\linewidth}
            \begin{enumerate}
                \item Représentation en bande passante.
            \end{enumerate}
        \end{minipage}
    \end{frame}

    \section{Résolution par l'étude des Multiflots}
    \subsection{Quelques Définitions}
    \begin{frame}\frametitle{\underline{Résolution par l'étude des Multiflots} \\ \small Quelques Définitions}
        \textbf{Définition 1:} Un flot représente l’acheminement d’un flux de matière depuis une source s vers une destination t.
        \\
    \vspace{2cm}
        \textbf{Définition 2:} Un multiflot consiste à faire cohabiter plusieurs flots sur le réseau de sorte que
        la somme des flots passant sur un arc soit inférieure à la capacité.
    \end{frame}

    \subsection{Problème de Multiflot Maximal}
    \begin{frame}\frametitle{\small Problème de Multiflot Maximal}
        On se munit ici d'un graphe $G(X,U)$ avec les sommets de $X$ numérotés de 1 à $n$.
        \\
        Dans $G$ un flot est définie par $\phi = (\phi_{1,2}, \phi_{2,3},..,\phi_{h-1,h}) \in \mathbb{R}^n$
        \\
        On pose A la matrice d'incidence Sommet-Arc.
        \\
        \vspace{0.5cm}
        \underline{Exemple:}
        $A = \[\left (
        \begin{array}{cccc}
            0 & 1 & 1 & 0 \\
            1 & 0 & 1 & 0 \\
            1 & 1 & 0 & 1 \\
            0 & 0 & 1 & 0 \\
        \end{array}
        \right )\]$.
        \\
        \vspace{0.5cm}
        On pose $p$ le nombre de flots distincts à router sur le graphe $G$.
        \\
        On pose aussi pour tout k appartenant à $[1,p]$, $\phi_{i,j}^{k}$ la quantité de flux qui passe sur l'arc (i,j)
        pour le flot k.
    \end{frame}

    \begin{frame}
        Ainsi le problème de Multiflot Maximal dont l'idée revient à trouver une distribution des flots qui maximise leur somme.
        C'est à dire à maximiser $Z = z^{1} + \cdot \cdot \cdot + z^{p}$. En maximisant cette quantité,
        on approche une solution convenable de la répartition des flux.
        \\
        Donc:
        \\
        \vspace{0,5cm}
        $(MM)
        \left\{
        \begin{array}{ll}
            \max(Z = z^{1} + \dots + z^{p}) \\
            \text{Sous les contraintes:} \\
            \forall k \in [1,p], A \cdot \phi^k = z^k \cdot b^k \\
            \forall (i,j) \in U, \displaystyle \sum_{k=0}^{p} \phi_{i,j}^{k} \le c_{i,j} \\
        \end{array}$
        \\
        \vspace{0.5cm}

        Cependant il est assez difficile de trouver une solution de (MM).
        Garg, vazirani et Yannakakison ont proposé un algorithme donnant une solution approchée à ce problème.\\
        Mais n'ayant rien trouvé dessus, on va donc s'intéresser à la compatibilité d'une solution donnée.
    \end{frame}

    \subsection{Problème de Multiflot Compatible}
    \begin{frame}\frametitle{\small Problème de Multiflot Compatible}
        La demande à satisfaire pour chaque flot et à présent une donnée fixée,
        cela correspond à la bande passante garantissant une bonne communication entre deux personnes. \\
        On pose alors, pour tout $k$ appartenant à $[1,p]$, $d^{k}$ la demande à satisfaire pour le couple $(s_{k}, t_{k})$. \\
        Ainsi vérifier si une distribution est compatible revient à vérifier: \\
        \vspace{0.3cm}
         $(MC)
        \left\{
        \begin{array}{ll}
            \forall k \in [1,p], A \cdot \phi^k = d^k \cdot b^k \\
            \forall (i,j) \in U, \displaystyle \sum_{k=0}^{p} \phi_{i,j}^{k} \le c_{i,j} \\
        \end{array}$
    \end{frame}

    \section{Etude du problème de Multiflot compatible}

    \subsection{Algorithme de sous gradient}
    \begin{frame}\frametitle{\underline{Etude du problème de Multiflot compatible} \\ \small Algorithme de sous gradient}
        \begin{enumerate}
            \item Choisir une solution de départ $\pi^{0}$. Exemple : \vspace{0.2cm} $\pi^{0} = \frac{1}{n} (1,1,\dots,1)$
            \item Répéter tant que condition d'arret non vérifié:
            \begin{enumerate}
                \item À l’étape courante t, $\pi^{t}$ est la solution courante. Déterminer le vecteur
                $R^{\sigma_{*}}$ défini par: \vspace{0.2cm} $\pi^{t} \cdot R^{\sigma_{*}}
                = \displaystyle \min_{\sigma \in \mathbb{R}} \left\{\pi^{t} \cdot R^{\sigma} \right\}$
                \item Si $L(\pi^{t}) \ge 0$ alors aucune solution n'est admissible.
                \item Sinon calculer $D^{t} = R^{\sigma_{*}} - c$, un sous gradient de $L$ en $\pi^{t}$, puis $\pi^{t+1}$,
                projection de $\lambda_{t}D^{t}$ sur S
                 \item $t \leftarrow t+1$
            \end{enumerate}
        \end{enumerate}
    \end{frame}

    \subsection{Quelques résultats}
    \begin{frame}\frametitle{\small Quelques résultats}
        \vspace{2cm}
        \begin{enumerate}
            \item Le programme dans le cas d'existence tourne en boucle infini
            \item Le programme converge vers un optimum du simplexe dans le cas d'existence d'une distribution compatible
        \end{enumerate}
    \end{frame}

    \begin{frame}
        \begin{minipage}{0.48\linewidth}
        \vspace{1.4cm}
        \begin{tikzpicture}
            \path
            (0,0) node[draw,shape=circle](0){0}
            (1,1) node[draw,shape=circle](1){1}
            (1,-1) node[draw,shape=circle](2){2}
            (-1,-1) node[draw,shape=circle](3){3}
            (-1,1) node[draw,shape=circle](4){4}
            (2,0) node[draw,shape=circle](5){5}
            (0,-2) node[draw,shape=circle](6){6}
            (-2,0) node[draw,shape=circle](7){7}
            (0,2) node[draw,shape=circle](8){8}
            \footnotesize
            \draw(8)--(1)node[midway,above,sloped]{$1000$};
            \draw(8)--(4)node[midway,above,sloped]{$1000$};
            \draw(4)--(0)node[midway,above,sloped]{$100$};
            \draw(1)--(0)node[midway,above,sloped]{$100$};
            \draw(4)--(7)node[midway,above,sloped]{$1000$};
            \draw(1)--(5)node[midway,above,sloped]{$100$};
            \draw(7)--(3)node[midway,above,sloped]{$100$};
            \draw(0)--(3)node[midway,above,sloped]{$100$};
            \draw(0)--(2)node[midway,above,sloped]{$100$};
            \draw(5)--(2)node[midway,above,sloped]{$100$};
            \draw(3)--(6)node[midway,above,sloped]{$100$};
            \draw(2)--(6)node[midway,above,sloped]{$10000$};
        \end{tikzpicture}
        \end{minipage}
        \hfill
        \begin{minipage}{0.48\linewidth}
            \begin{enumerate}
                \item On reprend cet exemple à 9 antennes pour tester l'algorithme
            \end{enumerate}
        \end{minipage}
    \end{frame}

    \begin{frame}
        \vspace{0.5cm}
        \includegraphics[scale = 0.4]{Pas de solutions}
        \vspace{0.5cm}
        \begin{enumerate}
            \item Pour cette demande de communication  et en fixant la bande passante pour chacun à 20, on a pas de solution
        \end{enumerate}
    \end{frame}

    \begin{frame}
        \vspace{0.5cm}
        \includegraphics[scale = 0.4]{Solution -25}
        \vspace{0.5cm}
        \begin{enumerate}
            \item Pour cette demande de communication  et en fixant la bande passante pour chacun à 15, on a une solution admissible
        \end{enumerate}
    \end{frame}

    \begin{frame}
        \vspace{0.5cm}
        \includegraphics[scale = 0.4]{Solution -10}
        \vspace{0.5cm}
        \begin{enumerate}
            \item Pour cette demande de communication  et en fixant la bande passante pour chacun à 15, on a encore une solution admissible
            \item L'optimum de la fonction L est plus proche de 0
        \end{enumerate}
    \end{frame}

    \begin{frame}

    \end{frame}

    \section{Annexe}

    \subsection{Formulation arcs-chemins du problème de Multiflot}
    \begin{frame}\frametitle{\underline{Annexe} \\ \small Formulation arcs-chemins du problème de Multiflot}
        On peut decomposer chacun des flots $\phi^{k}$ en somme de flots élémentaires s'écoulant sur
        des chemins élémentaires entre $s_{k}$ et $t_{k}$.
        \\
        On pose $p(k)$ le nombre de chemins élémentaires distincts entre $s_{k}$ et $t_{k}$ dans $G$,
        et pour $j = 1,\dots \dots,p(k)$, notons $Q^{k}_{j}$ le j-ième chemin de la liste.
        \\
        On definie le vecteur caractéristique de $Q^{k}_{j}$ par:
        \\
        \vspace{0.2cm}
        $(p_{j}^{k})_{u} = +1$ Si $u \in Q^{k}_{j}$ \\
        $(p_{j}^{k})_{u} = O$ Sinon.
        \\
        \vspace{0.2cm}
        Enfin on définie $x_{j}^{k}$ (prenant comme valeur $0$ ou $d^{k}$),
        la quantité de flot $k$ circulant sur la chemin $Q^{k}_{j}$.
    \end{frame}

    \begin{frame}
        Avec cette formulation, le problème de Multiflots compatible devient:
        \\
        \vspace{1cm}
        \centering
         $(MC')
        \left\{
        \begin{array}{ll}
            \text{Trouver les variables} \hspace{0.1cm} x_{j}^{k} \\
            \text{Vérifiant:} \\
            \forall k \in [1, p], \displaystyle \sum_{j=0}^{p(k)} x_{j}^{k} \ge d^{k} \\
            \displaystyle \sum_{k=0}^{p} \displaystyle \sum_{j=0}^{p(k)} p_{j}^{k} \cdot x_{j}^{k} \le c \hspace{2cm}
             C = (c_{1}, \dots, c_{n})^{T}
        \end{array}$
    \end{frame}

    \subsection{Condition d'existence d'une solution}
    \begin{frame} \frametitle{\small Condition d'existence d'une solution}
        on ré-ecrit $(MC')$ par : \\
        \vspace{0.4cm}
         $(MC')
        \left\{
        \begin{array}{ll}
            \text{Trouver les variables} \hspace{0.1cm} x_{j}^{k} \\
            \text{Vérifiant:} \\
            \forall k \in [1, p], - \displaystyle \sum_{j=0}^{p(k)} x_{j}^{k} \le - d^{k} \hspace{1cm} (3.1) \\
            \displaystyle \sum_{k=0}^{p} \displaystyle \sum_{j=0}^{p(k)} p_{j}^{k} \cdot x_{j}^{k} \le c \hspace{2cm} (3.2)
        \end{array}$
    \end{frame}
    
    \begin{frame}
        \underline{Lemme de Farkas et Minkowski:} \\
        Soit $A$ une matrice réelle $(m \cdot n)$ et $b$ $m$-vecteur.
        Le système d’inéquations linéaires : $Ax \le b$ a une solution non négative $(x \ge 0)$ si et seulement si : \\
        \vspace{0.2cm}
        \centering
        \boxed{$\forall y \in \mathbb{R}^{m}, y^{T} \cdot A \ge 0 \Rightarrow y^{T} \cdot b \ge 0$}
        \\
        \vspace{1cm}
        \begin{itemize}
            \item Associons $\pi_{u} \ge 0$ un multiplicateur de la contrainte $(3.1)$
            \item Associons $\mu^{k} \ge 0$ un multiplicateur de la contrainte $(3.2)$
            \item $\forall k \in [1,p], P^{k} = (  p_{1}^{k},...,p_{p(k)}^{k})$
        \end{itemize}
    \end{frame}

    \begin{frame}
        Grace au lemme de Farkas et Minkowski on a: \\
        \vspace{0.2cm}
        \boxed{
        Conditions
        $(C)
        \left\{
        \begin{array}{ll}
            \forall \pi \ge 0, \forall \mu \ge 0 \\
            \forall k \in [1, p], \pi \cdot P^{k} - \mu^{k} \ge 0 \Rightarrow \pi - \mu \cdot d \ge 0
        \end{array}$}
        \\
        \vspace{0.2cm}
        On observe que: \\
        \vspace{0.2cm}
        $\forall k \in [1, p], \pi \cdot P^{k} - \mu^{k} \ge 0
        \Rightarrow \mu^{k} \le \displaystyle \min_{1 \le j \le P(k)} \pi \cdot P_{j}^{k}$
        \\
        \vspace{0.2cm}
        Ainsi pour verifier (C), il faut et suffit que:
        \boxed{$\mu^{k} = \displaystyle \min_{1 \le j \le P(k)} \pi \cdot P_{j}^{k}$} \\
        \vspace{0.1cm}
        (Plus court chemin au sens de $\pi$ entre $s_{k}$ et $t_{k}$)
        \vspace{0.1cm}
        \\
        \underline{\bold{Théorème d'éxistence:}} \\
        une condition nécessaire et suffisante pour qu’il existe un multiflot compatible est que : \\
        $\forall \pi = (\pi_{1}, \pi_{2},\dots, \pi_{n})$: \\
        \centering
        \boxed{$\mu \cdot d \le \pi \cdot c$} \hspace{0.4cm} Avec $\mu = (\mu^{1},\dots, \mu^{p})$
    \end{frame}

    \begin{frame}
        \begin{itemize}
            \item On normalise le vecteur $\pi$
            \item On Recherche le maximum $L^{*}$ de la fonction $L(\pi)=\mu - \pi c$
            \item D'après le théoreme précédent:
            \begin{itemize}
                \item Si $L^{*} \le 0$ alors il existe une solution au problème
                \item Si $L^{*} \ge 0$ alors il n'éxiste pas de solution
            \end{itemize}
        \end{itemize}
        \underline{Definition:} \\
        Un routage est une application σ de l’ensemble des flots dans l’ensemble des chemins. \\
        Étant donné un routage σ, on appelle vecteur associé à $\sigma$ le vecteur : \\
        \\
        \centering
        $R^{\sigma} = \displaystyle \sum_{k=1}^{p} P_{\sigma(k)}^{k} \cdot d^{k}$
    \end{frame}

    \begin{frame}
        On considere le routage particulier $\sigma_{\pi}$ défini par: \\
        \vspace{0.2cm}
        \boxed{$\pi \cdot P_{\sigma_{\pi}(k)}^{k} = \displaystyle \min_{j = 1, \dots, p(k)} \pi \cdot P_{j}^{k}
        = \displaystyle \min_{\sigma} \pi \cdot P_{\sigma(k)}^{k}$}
        \vspace{1cm}
        \\
        La valeur de la fonction $L$ au point $\pi$ est alors :
        \\
        \vspace{0.2cm}
        \centering
        \boxed{$L(\pi) = \displaystyle \sum_{k=1}^{p} \mu^{k} \cdot d^{k} - \pi \cdot c =
        \displaystyle \sum_{k=1}^{p} \mu^{k} \cdot P_{\sigma_{\pi}(k)}^{k} - \pi \cdot c$}
        \\
        \vspace{0.5cm}
        D'ou:
        \\
        \vspace{0.5cm}
        \boxed{$L(\pi) = \pi \cdot (R^{\sigma_{\pi}} - c) = \displaystyle \min_{\sigma \in R} (R^{\sigma} - c)$}
    \end{frame}

    \begin{frame}
        Cette dernière expression montre que $L(\pi)$ est une fonction définie sur le simplexe:
        $S = \left\{ \pi | \pi \ge 0 \right\}$
    \end{frame}

    \subsection{Algorithme}

    \begin{frame}\frametitle{\small Algorithme}
        \includegraphics[scale=0.35]{algo 1}
    \end{frame}

    \begin{frame}
        \includegraphics[scale=0.35]{algo 2}
    \end{frame}

    \begin{frame}
        \includegraphics[scale=0.35]{algo 3}
    \end{frame}

    \begin{frame}
        \includegraphics[scale=0.35]{algo 4}
    \end{frame}

    \begin{frame}
        \includegraphics[scale=0.35]{algo 5}
    \end{frame}

    \begin{frame}
        \includegraphics[scale=0.35]{algo 6}
    \end{frame}

\end{document}